%%%%%%%%%%%%%%%%%%%%%%%%%%%%%%%%%%%%%%%%%%%%%%%%%%%%%%%%%%%%%%%%%%%%%%%%%%%%%%%
%%%%%%%%%%%%%%%%%%%%%%%%%%%%%%%%%%%%%%%%%%%%%%%%%%%%%%%%%%%%%%%%%%%%%%%%%%%%%%%
% %
% Title: Automatically generated Template for Dagstuhl Reports %
% Script used: abstracts-listing_tex.wml %
% originally developped by Tobias Maurer %
% 2005-02-15: Layout Template (by Jutta Huhse) %
% 2011-02-28: Layout adapted to Dagstuhl Reports (by Marc Herbstritt) %
% %
%%%%%%%%%%%%%%%%%%%%%%%%%%%%%%%%%%%%%%%%%%%%%%%%%%%%%%%%%%%%%%%%%%%%%%%%%%%%%%%
%
% This document requires the Dagstuhl Reports LaTeX style package dagrep.cls
% and corresponding graphic files.
% It can be downloaded from the following URL:
%
% http://www.dagstuhl.de/???
%
%%%%%%%%%%%%%%%%%%%%%%%%%%% head declarations %%%%%%%%%%%%%%%%%%%%%%%%%%%%%%%%%
%%%%%%%%%%%%%%%%%%%%%%%%%%%%%%%%%%%%%%%%%%%%%%%%%%%%%%%%%%%%%%%%%%%%%%%%%%%%%%%


%This is a template for producing reports for "Dagstuhl Reports".
%See dagrep.pdf for further information.

\documentclass[a4paper,UKenglish]{dagrep}
  %for A4 paper format use option "a4paper", for US-letter use option "letterpaper"
  %for british hyphenation rules use option "UKenglish", for american hyphenation rules use option "USenglish"
  %for section-numbered lemmas etc., use "numberwithinsect"

%\usepackage{todonotes}
\usepackage{microtype}%if unwanted, comment out or use option "draft"
\usepackage{url}
% ============================================================
%:Markup macros for proof-reading
\usepackage{ifthen}
\usepackage[normalem]{ulem} % for \sout
\usepackage{xcolor}
\newcommand{\ra}{$\rightarrow$}
\newboolean{showedits}
\setboolean{showedits}{true} % toggle to show or hide edits
%\setboolean{showedits}{false} % toggle to show or hide edits
\ifthenelse{\boolean{showedits}}
{
	\newcommand{\meh}[1]{\textcolor{red}{\uwave{#1}}} % please rephrase
	\newcommand{\ins}[1]{\textcolor{blue}{\uline{#1}}} % please insert
	\newcommand{\del}[1]{\textcolor{red}{\sout{#1}}} % please delete
	\newcommand{\chg}[2]{\textcolor{red}{\sout{#1}}{\ra}\textcolor{blue}{\uline{#2}}} % please change
	\newcommand{\nbe}[3]{
		{\colorbox{#3}{\bfseries\sffamily\scriptsize\textcolor{white}{#1}}}
		{\textcolor{#3}{\sf\small$\blacktriangleright$\textit{#2}$\blacktriangleleft$}}}
}{
	\newcommand{\meh}[1]{#1} % please rephrase
	\newcommand{\ins}[1]{#1} % please insert
	\newcommand{\del}[1]{} % please delete
	\newcommand{\chg}[2]{#2}
	\newcommand{\nbe}[3]{}
}
%
\newcommand\rA[1]{\nbe{Reviewer A}{#1}{cyan}}
\newcommand\rB[1]{\nbe{Reviewer B}{#1}{olive}}
\newcommand\rC[1]{\nbe{Reviewer C}{#1}{magenta}}
\newcommand\ANS[1]{\nbe{Response}{#1}{teal}}
% ============================================================
%:Box comments/edits
\usepackage[most]{tcolorbox}
\ifthenelse{\boolean{showedits}}
{
  \newtcolorbox{inserted}{%
       title=Inserted text:,
       colframe=blue,colback=blue!5!white,
       breakable,
       leftrule=0mm, 
       bottomrule=0mm,
       rightrule=0mm,
       toprule=0mm,
       arc=0mm, outer arc=0mm,
       oversize
  }
  \newtcolorbox{deleted}{%
       title=Deleted text:,
       colframe=red,colback=red!5!white,
       breakable,
       leftrule=0mm, 
       bottomrule=0mm,
       rightrule=0mm,
       toprule=0mm,
       arc=0mm, outer arc=0mm,
       oversize
  }
  \newtcolorbox{refactored}{%
       % title=Heavily modifed/refactored text:,
       title=Rewritten text:,
       colframe=blue,colback=red!5!white,
       breakable,
       leftrule=0mm, 
       bottomrule=0mm,
       rightrule=0mm,
       toprule=0mm,
       arc=0mm, outer arc=0mm,
       oversize
  }
}{
  \newenvironment{inserted}{}{}
  %\newenvironment{deleted}{ \begin{comment} }{ \end{comment} }
  \let\deleted\comment
  \newenvironment{refactored}{}{} 
}
% ============================================================
%:Put edit comments in a really ugly standout display
%\usepackage{ifthen}
\usepackage{amssymb}
\newboolean{showcomments}
\setboolean{showcomments}{true}
%\setboolean{showcomments}{false}
\newcommand{\id}[1]{$-$Id: scgPaper.tex 32478 2010-04-29 09:11:32Z oscar $-$}
\newcommand{\yellowbox}[1]{\fcolorbox{gray}{yellow}{\bfseries\sffamily\scriptsize#1}}
\newcommand{\triangles}[1]{{\sf\small$\blacktriangleright$\textit{#1}$\blacktriangleleft$}}
\ifthenelse{\boolean{showcomments}}
%{\newcommand{\nb}[2]{{\yellowbox{#1}\triangles{#2}}}
{\newcommand{\nbc}[3]{
 {\colorbox{#3}{\bfseries\sffamily\scriptsize\textcolor{white}{#1}}}
 {\textcolor{#3}{\sf\small$\blacktriangleright$\textit{#2}$\blacktriangleleft$}}}
 \newcommand{\version}{\emph{\scriptsize\id}}}
{\newcommand{\nbc}[3]{}
 \newcommand{\version}{}}
\newcommand{\nb}[2]{\nbc{#1}{#2}{orange}}
\newcommand{\here}{\yellowbox{$\Rightarrow$ CONTINUE HERE $\Leftarrow$}}
\newcommand\rev[2]{\nb{TODO (rev #1)}{#2}} % reviewer comments
\newcommand\fix[1]{\nb{FIX}{#1}}
\newcommand\todo[1]{\nb{TO DO}{#1}}
\newcommand\on[1]{\nbc{Oscar}{#1}{olive}} % add more author macros here
\newcommand\jv[1]{\nbc{Jurgen}{#1}{red}}
\newcommand\cg[1]{\nbc{Carol}{#1}{blue}}
\newcommand\jh[1]{\nbc{James}{#1}{brown}}
\newcommand\ck[1]{\nbc{Claude}{#1}{cyan}}
   \definecolor{darkgreen}{rgb}{0,0.6,0}
\newcommand\katznote[1]{\nbc{Dan}{#1}{darkgreen}} % add more author macros here

%\newcommand\XXX[1]{\nbc{XXX}{#1}{darkgray}}
%\newcommand\XXX[1]{\nbc{XXX}{#1}{gray}}
%\newcommand\XXX[1]{\nbc{XXX}{#1}{magenta}}
%\newcommand\XXX[1]{\nbc{XXX}{#1}{olive}}
%\newcommand\XXX[1]{\nbc{XXX}{#1}{orange}}
%\newcommand\XXX[1]{\nbc{XXX}{#1}{purple}}
%\newcommand\XXX[1]{\nbc{XXX}{#1}{red}}
%\newcommand\XXX[1]{\nbc{XXX}{#1}{teal}}
%\newcommand\XXX[1]{\nbc{XXX}{#1}{violet}}
% ============================================================


\bibliographystyle{plain}%the recommended bibstyle

% Preamble with header information
\subject{Report from Dagstuhl Seminar 16252}
\title{Perspectives Workshop: Engineering Academic Software}
\titlerunning{16252 -- Perspectives Workshop: Engineering Academic Software}%optional

\author[1]{Carole Goble}
  \affil[1]{University of Manchester, England \url{
mailto:carole.goble@manchester.ac.uk}}

\author[2]{James Howison}
  \affil[2]{The University of Texas at Austin, USA  \url{mailto:jhowison@ischool.utexas.edu}}

\author[3]{Claude Kirchner}
  \affil[3]{Inria, France   \url{mailto:claude.kirchner@inria.fr}}

\author[4]{Oscar Nierstrasz}
  \affil[4]{Stanford University, USA  \url{mailto:Claude.Kirchner@inria.fr}}

\author[5]{Jurgen J. Vinju}
  \affil[5]{Centrum Wiskunde \& Informatica, The Netherlands  \url{mailto:Jurgen.Vinju@cwi.nl}}

%Organizer macros:%%%%%%%%%%%%%%%%%%%%%%%%%%%%%%%%%%%%%%%%%%%%%%%%%%%%%
\seminarnumber{16252}
\semdata{\emph{20}.--\emph{24}.~\emph{June}, \emph{2016} -- \url{http://www.dagstuhl.de/16252}}
\subjclass{\emph{D.0. Software General}, \emph{}} % Cite, e.g., as B.3.3 Performance Analysis and Design Aids.} %mandatory
\keywords{\emph{Scientific Software, Data Science, Software Engineering}} % mandatory
% \additionaleditors{Tom Collector} %optional
%%%%%%%%%%%%%%%%%%%%%%%%%%%%%%%%%%%%%%%%%%%%%%%%%%%%%%%%%%%%%%%%%%%%%%%

%Dagstuhl editorial office macros:%%%%%%%%%%%%%%%%%%%%%%%%%%%%%%%%%%%%%
\volumeinfo%(easychair interface)
  {Carole Goble, James Howison,  Claude Kirchner, Oscar Nierstrasz, Jurgen Vinju}%editor names
  {5}%number of editors
  {Perspectives Workshop: Engineering Academic Software}%seminar title
  {1}%volume
  {1}%issue
  {1}%starting page number
\DOI{10.5362/DagRep.1.1.1}%(DagRep.<volume no>.<issue no>.<firstpage>)
%%%%%%%%%%%%%%%%%%%%%%%%%%%%%%%%%%%%%%%%%%%%%%%%%%%%%%%%%%%%%%%%%%%%%%%

\begin{document}

\maketitle

%------------------------------------------------------------------------
%------------------------------------------------------------------------
% this is a standard text with some seminar specific information
\begin{abstract}

\end{abstract}
%------------------------------------------------------------------------



% ==================================================
\section*{About the edit macros}

\on{Please use edit macros for comments you insert. See "edit-macros.tex" and add one for yourself if necessary.}

There are generic macros for \todo{stuff to do} and \fix{stuff to fix}.

There are macros for \ins{inserted text}, \del{deleted text}, and text to be changes \chg{from this}{to that}.


\begin{inserted}
There are also macros for blocks of text that have been inserted, deleted or refactored. These are useful to indicate proposals for changes to be checked by others in the pipeline.
\end{inserted}

% ==================================================
\section{Executive Summary}
\summaryauthor[Carole Goble, James Howison,  Claude Kirchner, Oscar Nierstrasz, Jurgen Vinju]{Carole Goble, James Howison,  Claude Kirchner, Oscar Nierstrasz, Jurgen Vinju}

\license

This perspective seminar brought together activists, experts and stakeholders on the subject of high quality software produced in an academic context\footnote{We include any software which is part of either research processes and/or output, while excluding more generic administrative software for research and education management.}. Our current dependence on software across the sciences is already significant, yet there are still more opportunities to be explored and risks to be overcome. The academic context is unique in terms of its personnel, its goals of exploring the unknown and its demands on quality assurance and reproducibility.

We refer to the IEEE Internet Computing article ``Better Software, Better Research''\footnote{\url{http://www.software.ac.uk/resources/publications/better-software-better-research}} which motivates the topic. In this seminar we took the following perspective of the research team which is in one of the following situations:
\begin{itemize}
\item They consume and/or produce software as output of the academic process;
\item They consume and/or produce software as an element of research methods;
\item They are in a combination of the above two situations.
\end{itemize}

Society is now in the tricky spot where several deeply established academic fields
(e.g. physics, biology, mathematics) are shifting towards total dependence on the fresh concepts of software, programming technology and software engineering methodology
which are backed only by young and rapidly evolving fields of research (computer science and software engineering).  Full accountability and even validity of software based research results are now duly being challenged.

With the outputs of this interactive and productive perspective seminar, we strive to contribute in a positive manner to the above challenges. We formulated taxonomies with definitions to clarify the domain, we co-authored concrete policy and process documents to improve the status and recognition of academic software development and academic software engineers, and finally we formulated a list of 18 concrete declaration of intent (``I will'' statements). This list was presented to the WSSSPE community~\cite{wssspe} to acquire feedback and it will be the backbone of the Dagstuhl Manifesto document we are editing. It serves to motivate change by  proposing policy changes with concrete actions and instilling positive attitudes towards academic software.


\textbf{The participants of the seminar} had three major ``blood-types''. The first group are active and visible members of the global academic software engineering community. They represent (formal) institutions such as the Software Sustainability Institute, the Software Carpentry Foundation, and eScience and datascience centers from across the globe. The second group contributed researchers in empirical software engineering, with a specific eye on studying the principles and practises of academic software engineering. The final group contributed researchers as an audience: software engineering researchers with a long experience in engineering software for software itself or software for specific academic research fields.

We found that without exception the participants were motivated and able to actively contribute to the proceedings of the seminar; the mix of people proved (accidentally\footnote{Most of the invitees did not belong to the computer science community and therefore invitations to the seminar were not immediately recognised as the opportunity which computer scientists know it is. The invitation process was therefore long and with a semi-accidental outcome.}) to be well balanced. To attest to this we've selectively listed three (paraphrased) verbal statements here:
\begin{itemize}
\item ``The seminar was a transformational experience for me	; I've learned an entire new perspective on my field and I intend to apply the insights in my daily practise.''
\item ``I had an epiphany yesterday after dinner; now I understand how to connect the datascience research at my university to the computer science department.''
\item ``Before the seminar I had no idea so many initiatives were already underway in [improving] academic software engineering; this has given my understanding of the challenges a real boost and I know what the some of the next steps to take are.''
\end{itemize}

\textbf{The schedule of the seminar} was designed to maximize interactive discussion and working towards tangible outputs. Key points were: to start the day with inspiring presentations to set the stage, then to have at least 40\% of the day time allocated to free discussion time, and to explicitly share successes (output) of each day's breakout groups at the end with the whole group.

The seminar started on Monday with a quick and tightly timed round of 2 minute personal introductions. Otherwise on Monday, Tuesday and Thursday the program was structured equally: in the morning we would have plenary presentations, which were interrupted ad-hoc by explorative discussions. These sessions were meant to bring everybody up-to-speed with ongoing and past initiatives. During and after lunch we used a board with sticky notes to define break-out groups. Each break-out group was centred around a specific discussion topic and (usually) a specific idea for an output document was associated with it. After coffee we would go back to the same break-out group to co-create the notes and lessons from each group (stored in a big shared online document). Between 17:00 and 18:00 we came back together and harvested the results of each breakout group with the others. People could and would freely switch between breakout groups but this was not a common thing.

On Wednesday we had an ``open-mic'' session with 8 presentations of around 10 minutes, sharing experiences and results, before we had a long walk in the surroundings. The organizers also designed an initial skeleton structure and ideas for the manifesto that day.

On Thursday afternoon and Friday morning we all worked together on our Dagstuhl Manifesto by first reworking our notes into the ideas around the manifesto, specifically a list of ``I will'' statements with references and motivation. Finally, Friday afternoon a small remaining group re-ordered the group's manifesto notes into a well structured list of 18 ``I will'' statements. Two of the organizers remained to continue to edit the current report and the manifesto document.

\textbf{Output documents} of the workshop are organized under the ``dagstuhleas'' organisation on Github: \url{https://github.com/dagstuhleas}. This currently features 6 draft documents, including the current report and (a) the manifesto, (b) Research Software Engineering Handbook (c) Literature Survey (d) Taxonomy on Software Credit Roles and (e) Software Award Proposal. Next to these documents R\&D project proposal was produced on measuring the impact of academic software.

\textbf{The remaining part of this document} summarizes the morning sessions by listing the abstracts of each talk, the afternoon breakouts by describing each topic and its results, and finally lists the research questions on the topic of engineering academic software we've collected.

\tableofcontents

\section{Overview of Talks}

These are the talks presented in the morning sessions of seminar, chronologically ordered.

%-------------------------------------------------------
%\newpage

\abstracttitle{Sustainable Software for Science}
\abstractauthor{Dan Katz (University of Illinois at Urbana Champaign, US)}

Progress in scientific research is dependent on the quality and accessibility of research software at all levels. It is now critical to address many new challenges related to the development, deployment, maintenance, and sustainability of open-use research software: the software upon which specific research results rely.  Open-use software means that the software is widely accessible (whether open source, shareware, or commercial).  Research software means that the choice of software is essential to specific research results; using different software could produce different results.

In addition, it is essential that scientists, researchers, and students are able to learn and adopt a new set of software-related skills and methodologies. Established researchers are already acquiring some of these skills, and in particular, a specialized class of software developers is emerging in academic environments who are an integral and embedded part of successful research teams. WSSSPE provides a forum for discussion of these challenges, including both positions and experiences, and a forum for the community to assemble and act.

This talk focuses on the Third Workshop on Sustainable Software for Science: Practice and Experiences (WSSSPE3). It summarized the discussions, future steps, organization, and status of a set of self-organized working groups on topics including developing pathways to funding scientific software; constructing useful common metrics for crediting software stakeholders; identifying principles for sustainable software engineering design; reaching out to research software organizations around the world; and building communities for software sustainability. Some of these groups have executed these activities that they scheduled, some have in part, and others have not.  A point of discussion is why these groups came to these points, and how the WSSSPE community can encourage groups to act.

\abstracttitle{Supporting Research Software Engineering}
\abstractauthor[]{Mike Croucher (University of Sheffield, UK)}

This talk detailed the support for Research Software Engineering and the job of Research Software Engineer at University of Sheffield. \jv{abstract by Mike is forthcoming}

\abstracttitle{Sustainability Design}
\abstractauthor[]{Christoph Becker (University of Toronto, CA)}

Sustainability - the "capacity to endure" - has emerged as a challenge with transformative impact on many disciplines and professions, including software engineering. It requires simultaneous consideration of at least five dimensions: environmental resources, social and individual well-being, economic prosperity, and long-term technical viability. This requires a cross-disciplinary approach to research and design that moves beyond narrow-minded solutionism to emphasize an appreciation of ‘wicked problems’ over a focus on puzzles and pieces; systems thinking over computational problem solving; and an integrated understanding of socio-technical systems. These shifts do not come easy, and for most systems, the hidden sustainability effects of past decisions in systems design are unknown. We can call this a system's 'sustainability debt'.

In this talk, I describe how synergies across a range of disciplines united by the need for new design approaches focused on sustainability led to the Karlskrona Manifesto for Sustainability Design. I characterize principles of sustainability design and the key influence of requirements activities on the sustainability debt of a system under design. I present recent efforts to develop this area of research, including an interview study of software professionals, as a starting point to a discussion of barriers and opportunities for sustainability design research.

See also: \url{www.sustainabilitydesign.org}

\abstracttitle{What We Have Learned About Using Software Engineering Practices in Scientific Software}
\abstractauthor[Jeffrey Carver]{Jeffrey Carver (University of Alabama, US)}

The increase in the importance of Scientific Software motivates the need to identify and understand which software engineering (SE) practices are appropriate. Because of the uniqueness of the scientific software domain, existing SE tools and techniques developed for the business/IT community are often not efficient or effective. Appropriate SE solutions must account for the salient characteristics of the scientific software development environment. To identify these solutions, members of the SE community must interact with members of the scientific software community. This presentation will discuss the findings from a series of case studies of scientific software projects, an ongoing workshop series, and the results of interactions between my research group and scientific software projects.

\abstracttitle{Lessons from the YT project}
\abstractauthor{Matthew Turk}

In this talk, I describe the engineering practices, both social and
technical, around the \href{yt project}{http://yt-project.org}.  I describe the
positive aspects and the failure modes, and how we have attempted to
route around these failure modes.

\abstracttitle{Software as Academic Output}
\abstractauthor[Caroline Jay, Robert Haines]{Caroline Jay and Robert Haines (University of Manchester, UK)}

Software is now considered to be an output of academic research in its own right: venues such as SoftwareX and the Journal of Open Research Software highlight it as a primary contribution, and the UK Research Council includes a software category in the ResearchFish\footnote{\url{http://www.researchfish.com}} application used to collect the outcomes of research projects. This phenomenon is still fairly recent, however, and two questions arise when trying to determine the validity of---or, arguably, requirement for---software as a product of the research process: when should it be considered an output and; what form should that output take?

To determine when software should be considered an output, we must consider its role in the research process. Is it a tool for supporting the work, or does it represent the research itself? To a computer scientist in the field of workflow management, the software would be considered a direct output, integral to the research. To a biologist, this same software would be considered a tool: useful for analyzing results, but not in itself an output of the research. For a bioinformatician both using and developing the tool, the answer is somewhere in the middle: whilst the core research may be in the life science domain, the modifications made to the tool as a result of this work could also be considered an output, advancing workflow management.

If the software is integral to the research---and therefore a potential output---what form should that output take? The FAIRDOM project\footnote{\url{http://fair-dom.org}} supports computational research that is FAIR: Findable, Accessible, Interoperable, Reusable. We suggest a modified version of these principles can be usefully applied to software too: it should be Findable, Accessible, Reusable and Extensible. To be findable, software must be searchable and discoverable by others, preferably via a persistent identifier. Accessible software can be viewed and downloaded by others. Reusable software can be re-run, potentially with other input data. Finally, Extensible software can be modified or extended to deal with new situations; to achieve this, the source code should be available.

The FARE principles are a starting point for defining best practice, or the `gold standard' for academic software outputs. An exemplar of the application of these principles is described in the authors' recent paper, `ABC: Using Object Tracking to Automate Behavioural Coding'~\cite{apaolaza_2016}, published at the 2016 ACM CHI conference. The source code is openly available on Github, making it accessible and extensible, and both this and the software environment (in a Docker container), are identified by DOIs, making the software findable and reusable.

Following the FARE principles will help to ensure that software intended to be a primary output of research is fit for purpose. Applying them in any situation where software is developed as part of research---whether it is considered a primary output or not---is also recommended, to help ensure that the resulting research is robust and reproducible.

\abstracttitle{Software Heritage}
\abstractauthor{Claude Kirchner (INRIA, FR)}

\jv{I wrote this, is it ok Claude?} In this talk I introduce \url{http://www.softwareheritage.org}, in the week just before its ``grand depart''.  
A quote from the website explains the concept and the goals:
\begin{quote}
``Software Heritage collects and preserves software in source code form, because software embodies our technical and scientific knowledge and humanity cannot afford the risk of losing it.
Software is a precious part of our cultural heritage. We curate and make accessible all the software we collect, because only by sharing it we can guarantee its preservation in the very long term.''
\end{quote}

\abstracttitle{Software Metadata: Describing ``dark software'' in Geosciences}
\abstractauthor[]{Daniel Garijo (Technical University of Madrid, ES)}

In this talk I provide an overview of the current state of the art for software description in geosciences, along with our approach to facilitate this task in OntoSoft, a distributed semantic registry for scientific software. Three key aspects of OntoSoft are: a software metadata ontology designed for scientists, a distributed approach to software registries that targets communities of interest, and metadata crowdsourcing through access control. Software metadata is organized using the OntoSoft ontology, designed to support scientists to share, document, and reuse software, and organized along six dimensions: identify software, understand and assess software, execute software, get support for the software, do research with the software, and update the software.

\abstracttitle{Organising a research team around the research software around the research team in software engineering}
\abstractauthor[]{Jurgen J. Vinju (Centrum Wiskunde \& Informatica, NL)}

This was about motivation, experiences and lessons learned around the SWAT research group at CWI and its core product/platform/framework/language used for doing research and transfer, \url{http://www.rascal-mpl.org} which is hosted from the open-source organisation \url{http://www.usethesource.io}.

\abstracttitle{Software Citation --- Principles, Discussion, and Metadata}
\abstractauthor[]{Dan Katz (University of Illinois at Urbana Champaign, US)}

This talk has practical info on how to semi-automatically take code on GitHub and publish it on Zenodo, obtaining a DOI that can then be cited can be found at \url{https://guides.github.com/activities/citable-code/}

\jv{adapted from Alice's blog; ok Dan?} I present an overview of work done by the Force11 Software Citation Working Group; including rationales for citing software, information on the WSSSPE and Force11 groups involved in developing software citation principles and the process used to develop them, and then the six principles, which focus on the importance of software, the need to credit and attribute the contributions software makes to research and to be able to uniquely identify software in a persistent and specific way, and that citations should enable access to the software and associated information about the software that informs its use. 

We bring up a number of the ongoing discussions at the WSSSPE and Force11 working groups and their determinations, such as what software to cite, how to uniquely identify software, that peer-review of software is important but not required for citation, and how publishers can help.

\abstracttitle{Katy Kuksenok (University of Washington - Seattle, US)}
\abstractauthor[]{Best Practices by Any Other Name}

\jv{adapted from Alice's blog, ok Katy?} This talk looked at intersections of the technical, social, and cognitive aspects of software engineering in research, and asked how the available community and skill resources could be leveraged. It brought together various elements brought up through the workshop so far, including different roles that had been identified, the need for software engineers to learn from scientists just as we hope researchers learn software engineering practices, and overcoming communications barriers.

\abstracttitle{ACL: restoring reproducibility --- Making scientist software discoverable}
\abstractauthor[]{Alice Allen (University of Maryland - College Park, US)}

\jv{adapted from your blog Alice, ok?}  I presented an overview of the ASCL, its history and a few of the changes to our infrastructure, the lessons we learned from looking at what other astro code registries and repositories had done and what we did with those lessons, and some of the impact we have on the community. 

\abstracttitle{A Short (and Probably Incomplete) History of Research Software Engineers in the UK}
\abstractauthor[Robert Haines]{Robert Haines (University of Manchester, UK)}

\begin{quote}
``\textit{Before software can be reusable it first has to be usable}''---Ralph Johnson, University of Illinois at Urbana-Champaign.
\end{quote}

A growing number of people in academia combine expertise in programming with an intricate understanding of research. Although this combination of skills is extremely valuable, these people lack a formal place in the academic system; they are not academics with a personal research agenda. This means there is no easy way to recognize their contribution, to reward them, or to represent their views.

One of the largest obstacles to overcome in recognizing this group of people is that they are often ``hiding'' in their institutions under a myriad different job titles and roles: Post-Doc, Research Associate, System Administrator, Computer Officer, and so on. In the instance of Post-Docs and Research Associates it is often the case that these people suddenly find that they have written too much code, and not enough papers, and so they fall foul of the usual metrics used to evaluate them for promotion. Being the person in the lab who ``knows about computers'' can be detrimental to your career.

These topics came up frequently at the Software Sustainability Institute's Collaborations Workshop in 2012. At this ``unconference'' style event a number of us repeatedly found ourselves in sessions discussing career paths, credit, recognition, metrics and reward, for those of us working in academia, who weren't academics. Without a name, it is difficult for people to rally around a cause, so we created the term Research Software Engineer (RSE) to describe the intersection of ``The Craftsperson and the Scholar''\footnote{\url{http://www.software.ac.uk/blog/2012-11-09-craftsperson-and-scholar}}. RSEs are facilitative, supportive and collaborative; part of the academic community and its institutional memory, providing continuity and stability for its academic software. We also created the UK Community of Research Software Engineers\footnote{\url{http://www.rse.ac.uk/}} (UKRSE) as a focal point for our future campaigns and the Institute made the promotion of the RSE job role a cornerstone of their policy, lobbying the UK Government and Research Councils for RSEs to be recognized at a high level.

Since 2012 the RSE job role has gained traction in a number of institutions and has been endorsed by the Engineering and Physical Sciences Research Council in the UK (EPSRC). There are groups employing RSEs in University College London, the University of Manchester, the University of Cambridge, the University of Southampton and the University of Sheffield, with more in the process of being set up all the time. The EPSRC has funded seven RSE Fellowships, who have a remit to develop and support software and users of software, and has also funded a network of RSE leaders\footnote{\url{http://gow.epsrc.ac.uk/NGBOViewGrant.aspx?GrantRef=EP/N028902/1}} to further build the community of RSEs and develop the next round of Fellowships.

This year we are holding the first RSE Conference in Manchester, UK. It will be the first conference to focus exclusively on the issues that affect people who write and use software in research and will include both presented talks and workshops where new tools and techniques will be taught. The conference is oversubscribed and has attracted much interest from the international community, so we look forward to expanding our community and building links with colleagues all over the world in the near future.

\abstracttitle{101companies - making a failing project succeed}
\abstractauthor[]{Ralf L\"ammel}

\jv{Ralf, I adapted this from Alice's blog, ok?} This talk is about the 101Companies project: a software chrestomathy, from chresto, meaning ``useful'' and `mathein, meaning ``to learn.'' 101 is a knowledge resource for technological space travel (between all kinds of technological spaces). It can serve to compare technologies, for programming education, and can serve as a playground for student projects. We discussed some of the challenges the project is having and some of the ways in which it is succeeding. 

\abstracttitle{UE eScience Institute Initiatives}
\abstractauthor[]{Cecilia Aragon (University of Washington - Seattle, US)}

Thanks in part to the recent popularity of the buzzword ``big data,'' it is now generally understood that many important scientific breakthroughs are made by interdisciplinary collaborations of scientists working in geographically distributed locations, producing and analyzing vast and complex data sets. The extraordinary advances in our ability to acquire and generate data in physical, biological, and social sciences are transforming the fundamental nature of science discovery across domains. Much of the research in this area, which has become known as data science or eScience, has focused on automated methods of analyzing data such as machine learning and new database techniques. Less attention has been directed to the human aspects of data science, including how to build interactive tools that maximize scientific creativity and human insight, and how to train, support, motivate, and retain the individuals with the necessary skills to produce the next generation of scientific discoveries.

In this talk, I will discuss history and ongoing initiatives at the UW eScience Institute, including opportunities to participate in the \$37.8M Moore/Sloan Data Science Environment at UW, UCB, and NYU, and speculate upon future directions for data science. In particular, I will discuss new initiatives at UW such as the eScience Incubator and the Data Science for Social Good program and will focus on results of ethnographic studies of our projects and future work in the Data Science Studies working group. Further, I will argue for the importance of a human centered approach to data science as necessary for the success of 21st century scientific discovery. I attest that we need to go beyond well-designed user interfaces for data science software tools to consider the entire ecosystem of software development and use: we need to study scientific collaborations interacting with technology as socio-technical systems, where both computer science and social science approaches are interwoven.

\abstracttitle{The Netherlands eScience Center}
\abstractauthor[Rob van Nieuwpoort]{Rob van Nieuwpoort (The Netherlands eScience Center (NLeSC), NL)}

The Netherlands eScience Center (NLeSC) is the Dutch national hub for
the development and application of domain overarching software and
methods for the scientific community. NLeSC develops crucial bridges
between increasingly complex modern e-infrastructures and the growing
demands and ambitions of scientists from across all disciplines. The
application of digitally enhanced scientific practices makes sure that
return can be achieved from scientific investments. In support of this
goal NLeSC funds and simultaneously funds and participates in
multidisciplinary projects, with academia and industry, with optimized
data-handling, efficient computing and big-data analytics at their
core.

NLeSC contributes exclusively to multidisciplinary projects with the
potential to deliver scientific excellence, in terms of breakthroughs
and in the realization of unique eScience methodologies. Many
organizational practices, such as our open call strategy and other
funding models ensure that new projects fulfill these criteria.

Apart from contributions to scientific publications in high-impact
scientific journals and conferences, NLeSC’s primary deliverables are
eScience instruments (e.g., software tools, workflows). Whilst the
instruments may include a domain specific component, primarily these
tools overarch multiple domains. The instruments are efficient,
calibrated, reliable and accessible and based on excellent standards
of code quality utilizing meta-data standards and software development
environments. Successful instruments are made publicly available as
part of NLeSC’s eScience technology platform (eStep) program. This
platform provides easy access to the developed tools and instruments
to the broader scientific community and industry alike. NLeSC also
shares non-scientific technical results, documentation and best
practices in the knowledge base that also is a part of eStep.

NLeSC also plays a key role in optimizing and disseminating the best
practices in the areas of software sustainability and
data-stewardship, including the need to engage with communities of
practices, data-publishers and related initiatives.

The rapid growth of data and computing initiatives risks unnecessary
fragmentation and duplication. NLeSC works with numerous partner
organizations, nationally and internationally, to identify common
challenges such as training and career support for eScientists, as
well as providing thought leadership on issues such as
data-stewardship and software sustainability. NLeSC is a joint
initiative of the Dutch national research council (NWO) and the Dutch
organization for ICT in education and research (SURF).



%\license
%\jointwork{Bry, Fran\c{c}ois;}
%\abstractref[]{} % [] - URL, {} - reference description (a la thebibliography)
%\abstractrefurl{} % preferrably DOI-based URL

\abstracttitle{On Impactstory and Debsy}
\abstractauthor{James Howison (University of Texas - Austin, US)}

There is a need to provide insight into the scientific software ecosystem~\cite{bogart_mapping_2015}: the set of projects, their software products, their authors, their dependencies, the papers describing them, and papers that have used the software to undertake science.  Such insight is useful for many players in the ecosystem, including end-users, software component producers, and ecosystem stewards (funders and senior scientists). I presented two systems that have attempted to gather and display this data: \href{depsy.org}{http://www.depsy.org} and the scientific software network map.

Depsy has gathered data from CRAN and PyPI as a starting point. They gather dependency and authorship information from those repositories. They identify the software in the literature using a fulltext keyword search for the name of the package. They are then able to calculate both direct mentions of each package and indirect mentions, using the Pagerank algorithm.  Depsy is produced by ImpactStory (Heather Piowawar and Jason Priem) and described in this article \cite{singh_chawla_unsung_2016}.

The second system presented was the Scientific Software Network Map by Bogart, Howison, and Herbsleb. The Map is designed to be populated from different ecosystems' software repositories, current work includes data from two locations: R scripts on github and data from the Texas Advanced Computing Center gathered about jobs submitted to their supercomputing infrastructure.  The interface uses d3 for the visualizations, and pyramid, mongo and jinja for the web and database framework. Maps are designed to directly address the needs of scientific software producers and stewards for usage-related information about packages. The tool's features include a usage graph over time, a filterable/sortable list of packages, a ``co-usage'' graph showing what packages were used together, and a listing of external software (e.g. end-user scripts and packages under development) that depend on each package. The co-usage graph could be used to identify previously unknown clusters of packages and to bring their developers together.

\abstracttitle{The OSSMETER platform}
\abstractauthor[]{Jurgen J. Vinju}

This briefly introduced the OSSMETER platform (\url{http://www.ossmeter.org}) for monitoring and comparing open-source software projects in terms of code, activity, issues and community. This open-source project can be used to monitor and assess projects and may be applicable to the domain of scientific software as well.

\abstracttitle{Code Work in Science: How it changes, and Why it matters how we talk about change}
\abstractauthor[]{Katie Kuksenok (University of Washington - Seattle, US)}

As software projects in science become more ambitious (and expensive in time, energy, and money), they increasingly require a language that recognizes and rewards the collective pursuit of uncertain possible worlds. I developed such a language, and I believe it would be useful to use it to articulate our current imagination, as a scholarly community, of the "perfect world" with regard to scientific programming. This articulation includes social and cognitive components of personalized "working environment." This allows me to include curiosity and excitement as much as panic and breakdown as causes, or opportunities, for small deliberate acts of change.

The conceptual framework is based on an 18-month interview and observation study of "code work" in science. "Code work," here, is used to include scripting, programming, software development, and other activities that span different projects but that demand common skills and aesthetics from the individuals undertaking them.

As attachments, I have included: (1) a LINK to a short blog post summary with figures (a good overview), (2) slides with a very short introduction, and (3) the final PDF of my dissertation (for the dedicated). Throughout the workshop, I intend to work on a reflection/summary of our discussion using the framework I have developed. In the Monday 2-minute introduction, I will introduce the framework, following the attached LINK (1). In the Wednesday open mic, I'll summarize in 10 minutes my "first-draft" take, which I would refine and contribute as part of the output document.

LINK (1): https://medium.com/hci-design-at-uw/code-work-in-science-how-it-changes-and-why-it-matters-how-we-talk-about-change-fecd33471b0

\abstracttitle{Restoring reproducibility: Making scientist software discoverable}
\abstractauthor[]{Alice Allen, University of Maryland (College Park, US)}

Taken from a presentation given at the National Library of Medicine (US), this Open Mic talk gives a quick overview of the Astrophysics Source Code Library (ASCL) and how it seeks to make research software used in astronomy more discoverable. It covers the changes in the ASCL over time and its impact.

The PowerPoint file has speaker notes that should mostly convey what I said in the presentation.

\section{Breakout sessions}

The afternoons of the seminar were dedicated to focused break-out groups. The groups were defined in a plenary discussion using a board with sticky notes. Everybody could propose topics. The topics were grouped on the board by topic similarity. Some groups continued over more than one day in order to arrive at a tangible results. All groups made notes into a single shared document. This same document was the source for the current report as well as the manifesto.

The breakout groups are detailed below in arbitrary order.

\abstracttitle{Research Software Project Typology}
\abstractauthor{Benoit Combemale, Jurgen Vinju, Robert Hirschfeldt, Ralf Laemmel, Daniel Garijo, Christoph Becker, Caroline Jay, Robert Haines, Cecilia Aragon}

Our goal here is to characterize academic software projects in a way that is meaningful for to the other ongoing discussions, such as measuring software output and acquiring funding for research software, and getting recognition of software outpu). The idea is that different types of projects require a different kind of treatment; e.g. when discussing their motivation, impact, costs, risks, strengths and weaknesses. We’d like this to be an ontology of sorts.

The break-out group lasted during two days and has produced a work-in-progress summarized at \url{https://w3id.org/softwareCredit}. The current status is that we are aligning the software credit ontology to DOAP and schema.org. We will soon list the crosswalk to codemeta terms.

\abstracttitle{Empirical study of software in conferences}
\abstractauthor{Jeffrey Carver, James Howison, Robert Haines, Caroline Jay, Kevin Crowston, Oscar Nierstrasz}

\abstracttitle{Examining sustainability for a particular project}
\abstractauthor{James Howison, Carole Goble}

\abstracttitle{Making the impact of software more visible}

\abstracttitle{Reviewing FORCE11 software citation principles}

\abstracttitle{Future Research directions}

\abstracttitle{Design of the Manifesto}
\abstractauthor{Claude Kirchner, Oscar Nierstrasz, James Howison, Katie Kuksenok, Jurgen Vinju}

\abstracttitle{Academic Software Engineering Roles Typology}
\abstractauthor{Daniel Garijo, Rob van Nieuwpoort, Jurgen Vinju}


\section{Research questions}


\bibliographystyle{plain}
\bibliography{report}
\end{document}
