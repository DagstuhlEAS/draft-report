%%%%%%%%%%%%%%%%%%%%%%%%%%%%%%%%%%%%%%%%%%%%%%%%%%%%%%%%%%%%%%%%%%%%%%%%%%%%%%%
%%%%%%%%%%%%%%%%%%%%%%%%%%%%%%%%%%%%%%%%%%%%%%%%%%%%%%%%%%%%%%%%%%%%%%%%%%%%%%%
% %
% Title: Automatically generated Template for Dagstuhl Reports %
% Script used: abstracts-listing_tex.wml %
% originally developped by Tobias Maurer %
% 2005-02-15: Layout Template (by Jutta Huhse) %
% 2011-02-28: Layout adapted to Dagstuhl Reports (by Marc Herbstritt) %
% %
%%%%%%%%%%%%%%%%%%%%%%%%%%%%%%%%%%%%%%%%%%%%%%%%%%%%%%%%%%%%%%%%%%%%%%%%%%%%%%%
% 
% This document requires the Dagstuhl Reports LaTeX style package dagrep.cls 
% and corresponding graphic files. 
% It can be downloaded from the following URL:
%
% http://www.dagstuhl.de/???
%
%%%%%%%%%%%%%%%%%%%%%%%%%%% head declarations %%%%%%%%%%%%%%%%%%%%%%%%%%%%%%%%%
%%%%%%%%%%%%%%%%%%%%%%%%%%%%%%%%%%%%%%%%%%%%%%%%%%%%%%%%%%%%%%%%%%%%%%%%%%%%%%%


%This is a template for producing reports for "Dagstuhl Reports".
%See dagrep.pdf for further information.

\documentclass[a4paper,UKenglish]{dagrep}
  %for A4 paper format use option "a4paper", for US-letter use option "letterpaper"
  %for british hyphenation rules use option "UKenglish", for american hyphenation rules use option "USenglish"
  %for section-numbered lemmas etc., use "numberwithinsect"

\usepackage{todonotes}
\usepackage{microtype}%if unwanted, comment out or use option "draft"
\usepackage{url}

\bibliographystyle{plain}%the recommended bibstyle

% Preamble with header information 
\subject{Report from Dagstuhl Seminar 16252}
\title{Perspectives Workshop: Engineering Academic Software}
\titlerunning{16252 -- Perspectives Workshop: Engineering Academic Software}%optional

\author[1]{Carole Goble}
  \affil[1]{University of Manchester, England \url{
mailto:carole.goble@manchester.ac.uk}}

\author[2]{James Howison}
  \affil[2]{The University of Texas at Austin, USA  \url{mailto:jhowison@ischool.utexas.edu}}
  
\author[3]{Claude Kirchner}
  \affil[3]{Inria, France   \url{mailto:claude.kirchner@inria.fr}}

\author[4]{Oscar Nierstrasz}
  \affil[4]{Stanford University, USA  \url{mailto:Claude.Kirchner@inria.fr}}

\author[5]{Jurgen J. Vinju}
  \affil[5]{Centrum Wiskunde \& Informatica, The Netherlands  \url{mailto:Jurgen.Vinju@cwi.nl}}

%Organizer macros:%%%%%%%%%%%%%%%%%%%%%%%%%%%%%%%%%%%%%%%%%%%%%%%%%%%%%
\seminarnumber{16252}
\semdata{\emph{20}.--\emph{24}.~\emph{June}, \emph{2016} -- \url{http://www.dagstuhl.de/16252}}
\subjclass{\emph{D.0. Software General}, \emph{}} % Cite, e.g., as B.3.3 Performance Analysis and Design Aids.} %mandatory
\keywords{\emph{Scientific Software, Data Science, Software Engineering}} % mandatory
% \additionaleditors{Tom Collector} %optional
%%%%%%%%%%%%%%%%%%%%%%%%%%%%%%%%%%%%%%%%%%%%%%%%%%%%%%%%%%%%%%%%%%%%%%%

%Dagstuhl editorial office macros:%%%%%%%%%%%%%%%%%%%%%%%%%%%%%%%%%%%%%
\volumeinfo%(easychair interface)
  {Carole Goble, James Howison,  Claude Kirchner, Oscar Nierstrasz, Jurgen Vinju}%editor names
  {5}%number of editors
  {Perspectives Workshop: Engineering Academic Software}%seminar title
  {1}%volume
  {1}%issue
  {1}%starting page number
\DOI{10.5362/DagRep.1.1.1}%(DagRep.<volume no>.<issue no>.<firstpage>)
%%%%%%%%%%%%%%%%%%%%%%%%%%%%%%%%%%%%%%%%%%%%%%%%%%%%%%%%%%%%%%%%%%%%%%%

\begin{document}

\maketitle

%------------------------------------------------------------------------
%------------------------------------------------------------------------
% this is a standard text with some seminar specific information
\begin{abstract}
 
\end{abstract}
%------------------------------------------------------------------------



\section{Executive Summary}
\summaryauthor[Carole Goble, James Howison,  Claude Kirchner, Oscar Nierstrasz, Jurgen Vinju]{Carole Goble, James Howison,  Claude Kirchner, Oscar Nierstrasz, Jurgen Vinju}

\license

This perspective seminar brought together activists, experts and stakeholders on the subject of high quality software produced in an academic context\footnote{We mean include any software which is part of the research processes and output, while excluding more generic administrative software for research and education management.}. Our current dependence on software across the sciences is already significant, yet there are still more opportunities to be explored and risks to be overcome. The academic context is unique in terms of its personnel, its goals of exploring the unknown and its demands on quality assurance and reproducibility. 

We refer to the IEEE Internet Computing article ``Better Software, Better Research''\footnote{\url{http://www.software.ac.uk/resources/publications/better-software-better-research}} which motivates the topic. In this seminar we took the following perspective of the research team which is in one of the following situations:
\begin{itemize}
\item They consume and/or produce software as output of the academic process;
\item They consume and/or produce software as an element of research methods;
\item They are in a combination of the above two situations.
\end{itemize}

Society is now in the tricky spot where several deeply established academic fields
(e.g. physics, biology, mathematics) are shifting towards total dependence on the fresh concepts of software, programming technology and software engineering methodology 
which are backed only by young and rapidly evolving fields of research (computer science and software engineering).  Full accountability and even validity of software based research results are now duly being challenged. 

With the outputs of this interactive and productive perspective seminar, we strive to contribute in a positive manner to the above challenges. We formulated taxonomies with definitions to clarify the domain, we co-authored concrete policy and process documents to improve the status and recognition of academic software development and academic software engineers, and finally we formulated a list of 18 concrete declaration of intent (``I will'' statements). This list is the backbone of the Dagstuhl Manifesto document~\cite{manifesto} we are editing. It serves to motivate change by  proposing policy changes with concrete actions and instilling positive attitudes towards academic software.
 

\textbf{The participants of the seminar} had three major ``blood-types''. The first group are active and visible members of the global academic software engineering community. They represent (formal) institutions such as the Software Sustainability Institute, the Software Carpentry Foundation, and eScience and datascience centers from across the globe. The second group contributed researchers in empirical software engineering, with a specific eye on studying the principles and practises of academic software engineering. The final group contributed researchers as an audience: software engineering researchers with a long experience in engineering software for software itself or software for specific academic research fields. 

We found that without exception the participants were motivated and able to actively contribute to the proceedings of the seminar; the mix of people proved (accidentally\footnote{Most of the invitees did not belong to the computer science community and therefore invitations to the seminar were not immediately recognised as the opportunity which computer scientists know it is. The invitation process was therefore long and with a semi-accidental outcome.}) to be well balanced. To attest to this we've selectively listed three (paraphrased) verbal statements here:
\begin{itemize}
\item ``The seminar was a transformational experience for me	; I've learned an entire new perspective on my field and I intend to apply the insights in my daily practise.''
\item ``I had an epiphany yesterday after dinner; now I understand how to connect the datascience research at my university to the computer science department.''
\item ``Before the seminar I had no idea so many initiatives were already underway in [improving] academic software engineering; this has given my understanding of the challenges a real boost and I know what the some of the next steps to take are.''
\end{itemize}

\textbf{The schedule of the seminar} was designed to maximize interactive discussion and working towards tangible outputs. Key points were: to start the day with inspiring presentations to set the stage, then to have at least 40\% of the day time allocated to free discussion time, and to explicitly share successes (output) of each day's breakout group at the end with the whole group. 

The seminar started on Monday with a quick and tightly timed round of 2 minute personal introductions. Otherwise on Monday, Tuesday and Thursday the program was structured equally: in the morning we would have plenary presentations, which were interrupted ad-hoc by explorative discussions. These sessions were meant to bring everybody up-to-speed with ongoing and past initiatives. During and after lunch we used a board with sticky notes to define break-out groups. Each break-out group was centred around a specific discussion topic and (usually) a specific idea for an output document was associated with it. After coffee we would go back to the same break-out group to co-create the notes and lessons from each group (stored in a big shared online document). Between 17:00 and 18:00 we came back together and harvested the results of each breakout group with the others. People could and would freely switch between breakout groups but this was not a common thing.

On Wednesday we had an ``open-mic'' session with 8 presentations of around 10 minutes, sharing experiences and results, before we had a long walk in the surroundings. The organizers also designed an initial skeleton structure and ideas for the manifesto that day.

On Thursday afternoon and Friday morning we all worked together on our Dagstuhl Manifesto by first reworking our notes into the ideas around the manifesto, specifically a list of ``I will'' statements with references and motivation. Finally, Friday afternoon a small remaining group re-ordered the group's manifesto notes into a well structured list of 18 ``I will'' statements. Two of the organizers remained until Saturday morning to continue to edit the current report and the manifesto document.

\tableofcontents

\section{Overview of Talks}

%-------------------------------------------------------
%\newpage
\abstracttitle{eScience Center}
\abstractauthor[Rob van Nieuwpoort]{Rob van Nieuwpoort (eScience Center, NL)}
%\license
%\jointwork{Bry, Fran\c{c}ois;}
%\abstractref[]{} % [] - URL, {} - reference description (a la thebibliography)
%\abstractrefurl{} % preferrably DOI-based URL

  
\section{Research questions}

\section{Breakout sessions}

\end{document}
