%%%%%%%%%%%%%%%%%%%%%%%%%%%%%%%%%%%%%%%%%%%%%%%%%%%%%%%%%%%%%%%%%%%%%%%%%%%%%%%
%%%%%%%%%%%%%%%%%%%%%%%%%%%%%%%%%%%%%%%%%%%%%%%%%%%%%%%%%%%%%%%%%%%%%%%%%%%%%%%
% %
% Title: Automatically generated Template for Dagstuhl Reports %
% Script used: abstracts-listing_tex.wml %
% originally developped by Tobias Maurer %
% 2005-02-15: Layout Template (by Jutta Huhse) %
% 2011-02-28: Layout adapted to Dagstuhl Reports (by Marc Herbstritt) %
% %
%%%%%%%%%%%%%%%%%%%%%%%%%%%%%%%%%%%%%%%%%%%%%%%%%%%%%%%%%%%%%%%%%%%%%%%%%%%%%%%
% 
% This document requires the Dagstuhl Reports LaTeX style package dagrep.cls 
% and corresponding graphic files. 
% It can be downloaded from the following URL:
%
% http://www.dagstuhl.de/???
%
%%%%%%%%%%%%%%%%%%%%%%%%%%% head declarations %%%%%%%%%%%%%%%%%%%%%%%%%%%%%%%%%
%%%%%%%%%%%%%%%%%%%%%%%%%%%%%%%%%%%%%%%%%%%%%%%%%%%%%%%%%%%%%%%%%%%%%%%%%%%%%%%


%This is a template for producing reports for "Dagstuhl Reports".
%See dagrep.pdf for further information.

\documentclass[a4paper,UKenglish]{dagrep}
  %for A4 paper format use option "a4paper", for US-letter use option "letterpaper"
  %for british hyphenation rules use option "UKenglish", for american hyphenation rules use option "USenglish"
  %for section-numbered lemmas etc., use "numberwithinsect"

%\usepackage{todonotes}
\usepackage{microtype}%if unwanted, comment out or use option "draft"
\usepackage{url}
% ============================================================
%:Markup macros for proof-reading
\usepackage{ifthen}
\usepackage[normalem]{ulem} % for \sout
\usepackage{xcolor}
\newcommand{\ra}{$\rightarrow$}
\newboolean{showedits}
\setboolean{showedits}{true} % toggle to show or hide edits
%\setboolean{showedits}{false} % toggle to show or hide edits
\ifthenelse{\boolean{showedits}}
{
	\newcommand{\meh}[1]{\textcolor{red}{\uwave{#1}}} % please rephrase
	\newcommand{\ins}[1]{\textcolor{blue}{\uline{#1}}} % please insert
	\newcommand{\del}[1]{\textcolor{red}{\sout{#1}}} % please delete
	\newcommand{\chg}[2]{\textcolor{red}{\sout{#1}}{\ra}\textcolor{blue}{\uline{#2}}} % please change
	\newcommand{\nbe}[3]{
		{\colorbox{#3}{\bfseries\sffamily\scriptsize\textcolor{white}{#1}}}
		{\textcolor{#3}{\sf\small$\blacktriangleright$\textit{#2}$\blacktriangleleft$}}}
}{
	\newcommand{\meh}[1]{#1} % please rephrase
	\newcommand{\ins}[1]{#1} % please insert
	\newcommand{\del}[1]{} % please delete
	\newcommand{\chg}[2]{#2}
	\newcommand{\nbe}[3]{}
}
%
\newcommand\rA[1]{\nbe{Reviewer A}{#1}{cyan}}
\newcommand\rB[1]{\nbe{Reviewer B}{#1}{olive}}
\newcommand\rC[1]{\nbe{Reviewer C}{#1}{magenta}}
\newcommand\ANS[1]{\nbe{Response}{#1}{teal}}
% ============================================================
%:Box comments/edits
\usepackage[most]{tcolorbox}
\ifthenelse{\boolean{showedits}}
{
  \newtcolorbox{inserted}{%
       title=Inserted text:,
       colframe=blue,colback=blue!5!white,
       breakable,
       leftrule=0mm, 
       bottomrule=0mm,
       rightrule=0mm,
       toprule=0mm,
       arc=0mm, outer arc=0mm,
       oversize
  }
  \newtcolorbox{deleted}{%
       title=Deleted text:,
       colframe=red,colback=red!5!white,
       breakable,
       leftrule=0mm, 
       bottomrule=0mm,
       rightrule=0mm,
       toprule=0mm,
       arc=0mm, outer arc=0mm,
       oversize
  }
  \newtcolorbox{refactored}{%
       % title=Heavily modifed/refactored text:,
       title=Rewritten text:,
       colframe=blue,colback=red!5!white,
       breakable,
       leftrule=0mm, 
       bottomrule=0mm,
       rightrule=0mm,
       toprule=0mm,
       arc=0mm, outer arc=0mm,
       oversize
  }
}{
  \newenvironment{inserted}{}{}
  %\newenvironment{deleted}{ \begin{comment} }{ \end{comment} }
  \let\deleted\comment
  \newenvironment{refactored}{}{} 
}
% ============================================================
%:Put edit comments in a really ugly standout display
%\usepackage{ifthen}
\usepackage{amssymb}
\newboolean{showcomments}
\setboolean{showcomments}{true}
%\setboolean{showcomments}{false}
\newcommand{\id}[1]{$-$Id: scgPaper.tex 32478 2010-04-29 09:11:32Z oscar $-$}
\newcommand{\yellowbox}[1]{\fcolorbox{gray}{yellow}{\bfseries\sffamily\scriptsize#1}}
\newcommand{\triangles}[1]{{\sf\small$\blacktriangleright$\textit{#1}$\blacktriangleleft$}}
\ifthenelse{\boolean{showcomments}}
%{\newcommand{\nb}[2]{{\yellowbox{#1}\triangles{#2}}}
{\newcommand{\nbc}[3]{
 {\colorbox{#3}{\bfseries\sffamily\scriptsize\textcolor{white}{#1}}}
 {\textcolor{#3}{\sf\small$\blacktriangleright$\textit{#2}$\blacktriangleleft$}}}
 \newcommand{\version}{\emph{\scriptsize\id}}}
{\newcommand{\nbc}[3]{}
 \newcommand{\version}{}}
\newcommand{\nb}[2]{\nbc{#1}{#2}{orange}}
\newcommand{\here}{\yellowbox{$\Rightarrow$ CONTINUE HERE $\Leftarrow$}}
\newcommand\rev[2]{\nb{TODO (rev #1)}{#2}} % reviewer comments
\newcommand\fix[1]{\nb{FIX}{#1}}
\newcommand\todo[1]{\nb{TO DO}{#1}}
\newcommand\on[1]{\nbc{Oscar}{#1}{olive}} % add more author macros here
\newcommand\jv[1]{\nbc{Jurgen}{#1}{red}}
\newcommand\cg[1]{\nbc{Carol}{#1}{blue}}
\newcommand\jh[1]{\nbc{James}{#1}{brown}}
\newcommand\ck[1]{\nbc{Claude}{#1}{cyan}}
   \definecolor{darkgreen}{rgb}{0,0.6,0}
\newcommand\katznote[1]{\nbc{Dan}{#1}{darkgreen}} % add more author macros here

%\newcommand\XXX[1]{\nbc{XXX}{#1}{darkgray}}
%\newcommand\XXX[1]{\nbc{XXX}{#1}{gray}}
%\newcommand\XXX[1]{\nbc{XXX}{#1}{magenta}}
%\newcommand\XXX[1]{\nbc{XXX}{#1}{olive}}
%\newcommand\XXX[1]{\nbc{XXX}{#1}{orange}}
%\newcommand\XXX[1]{\nbc{XXX}{#1}{purple}}
%\newcommand\XXX[1]{\nbc{XXX}{#1}{red}}
%\newcommand\XXX[1]{\nbc{XXX}{#1}{teal}}
%\newcommand\XXX[1]{\nbc{XXX}{#1}{violet}}
% ============================================================


\bibliographystyle{plain}%the recommended bibstyle

% Preamble with header information 
\subject{Report from Dagstuhl Seminar 16252}
\title{Perspectives Workshop: Engineering Academic Software}
\titlerunning{16252 -- Perspectives Workshop: Engineering Academic Software}%optional

\author[1]{Carole Goble}
  \affil[1]{University of Manchester, England \url{
mailto:carole.goble@manchester.ac.uk}}

\author[2]{James Howison}
  \affil[2]{The University of Texas at Austin, USA  \url{mailto:jhowison@ischool.utexas.edu}}
  
\author[3]{Claude Kirchner}
  \affil[3]{Inria, France   \url{mailto:claude.kirchner@inria.fr}}

\author[4]{Oscar Nierstrasz}
  \affil[4]{Stanford University, USA  \url{mailto:Claude.Kirchner@inria.fr}}

\author[5]{Jurgen J. Vinju}
  \affil[5]{Centrum Wiskunde \& Informatica, The Netherlands  \url{mailto:Jurgen.Vinju@cwi.nl}}

%Organizer macros:%%%%%%%%%%%%%%%%%%%%%%%%%%%%%%%%%%%%%%%%%%%%%%%%%%%%%
\seminarnumber{16252}
\semdata{\emph{20}.--\emph{24}.~\emph{June}, \emph{2016} -- \url{http://www.dagstuhl.de/16252}}
\subjclass{\emph{D.0. Software General}, \emph{}} % Cite, e.g., as B.3.3 Performance Analysis and Design Aids.} %mandatory
\keywords{\emph{Scientific Software, Data Science, Software Engineering}} % mandatory
% \additionaleditors{Tom Collector} %optional
%%%%%%%%%%%%%%%%%%%%%%%%%%%%%%%%%%%%%%%%%%%%%%%%%%%%%%%%%%%%%%%%%%%%%%%

%Dagstuhl editorial office macros:%%%%%%%%%%%%%%%%%%%%%%%%%%%%%%%%%%%%%
\volumeinfo%(easychair interface)
  {Carole Goble, James Howison,  Claude Kirchner, Oscar Nierstrasz, Jurgen Vinju}%editor names
  {5}%number of editors
  {Perspectives Workshop: Engineering Academic Software}%seminar title
  {1}%volume
  {1}%issue
  {1}%starting page number
\DOI{10.5362/DagRep.1.1.1}%(DagRep.<volume no>.<issue no>.<firstpage>)
%%%%%%%%%%%%%%%%%%%%%%%%%%%%%%%%%%%%%%%%%%%%%%%%%%%%%%%%%%%%%%%%%%%%%%%

\begin{document}

\maketitle

%------------------------------------------------------------------------
%------------------------------------------------------------------------
% this is a standard text with some seminar specific information
\begin{abstract}
 
\end{abstract}
%------------------------------------------------------------------------



% ==================================================
\section*{About the edit macros}

\on{Please use edit macros for comments you insert. See "edit-macros.tex" and add one for yourself if necessary.}

There are generic macros for \todo{stuff to do} and \fix{stuff to fix}.

There are macros for \ins{inserted text}, \del{deleted text}, and text to be changes \chg{from this}{to that}.


\begin{inserted}
There are also macros for blocks of text that have been inserted, deleted or refactored. These are useful to indicate proposals for changes to be checked by others in the pipeline.
\end{inserted}

% ==================================================
\section{Executive Summary}
\summaryauthor[Carole Goble, James Howison,  Claude Kirchner, Oscar Nierstrasz, Jurgen Vinju]{Carole Goble, James Howison,  Claude Kirchner, Oscar Nierstrasz, Jurgen Vinju}

\license

This perspective seminar brought together activists, experts and stakeholders on the subject of high quality software produced in an academic context\footnote{We mean include any software which is part of the research processes and output, while excluding more generic administrative software for research and education management.}. Our current dependence on software across the sciences is already significant, yet there are still more opportunities to be explored and risks to be overcome. The academic context is unique in terms of its personnel, its goals of exploring the unknown and its demands on quality assurance and reproducibility. 

We refer to the IEEE Internet Computing article ``Better Software, Better Research''\footnote{\url{http://www.software.ac.uk/resources/publications/better-software-better-research}} which motivates the topic. In this seminar we took the following perspective of the research team which is in one of the following situations:
\begin{itemize}
\item They consume and/or produce software as output of the academic process;
\item They consume and/or produce software as an element of research methods;
\item They are in a combination of the above two situations.
\end{itemize}

Society is now in the tricky spot where several deeply established academic fields
(e.g. physics, biology, mathematics) are shifting towards total dependence on the fresh concepts of software, programming technology and software engineering methodology 
which are backed only by young and rapidly evolving fields of research (computer science and software engineering).  Full accountability and even validity of software based research results are now duly being challenged. 

With the outputs of this interactive and productive perspective seminar, we strive to contribute in a positive manner to the above challenges. We formulated taxonomies with definitions to clarify the domain, we co-authored concrete policy and process documents to improve the status and recognition of academic software development and academic software engineers, and finally we formulated a list of 18 concrete declaration of intent (``I will'' statements). This list is the backbone of the Dagstuhl Manifesto document~\cite{manifesto} we are editing. It serves to motivate change by  proposing policy changes with concrete actions and instilling positive attitudes towards academic software.
 

\textbf{The participants of the seminar} had three major ``blood-types''. The first group are active and visible members of the global academic software engineering community. They represent (formal) institutions such as the Software Sustainability Institute, the Software Carpentry Foundation, and eScience and datascience centers from across the globe. The second group contributed researchers in empirical software engineering, with a specific eye on studying the principles and practises of academic software engineering. The final group contributed researchers as an audience: software engineering researchers with a long experience in engineering software for software itself or software for specific academic research fields. 

We found that without exception the participants were motivated and able to actively contribute to the proceedings of the seminar; the mix of people proved (accidentally\footnote{Most of the invitees did not belong to the computer science community and therefore invitations to the seminar were not immediately recognised as the opportunity which computer scientists know it is. The invitation process was therefore long and with a semi-accidental outcome.}) to be well balanced. To attest to this we've selectively listed three (paraphrased) verbal statements here:
\begin{itemize}
\item ``The seminar was a transformational experience for me	; I've learned an entire new perspective on my field and I intend to apply the insights in my daily practise.''
\item ``I had an epiphany yesterday after dinner; now I understand how to connect the datascience research at my university to the computer science department.''
\item ``Before the seminar I had no idea so many initiatives were already underway in [improving] academic software engineering; this has given my understanding of the challenges a real boost and I know what the some of the next steps to take are.''
\end{itemize}

\textbf{The schedule of the seminar} was designed to maximize interactive discussion and working towards tangible outputs. Key points were: to start the day with inspiring presentations to set the stage, then to have at least 40\% of the day time allocated to free discussion time, and to explicitly share successes (output) of each day's breakout group at the end with the whole group. 

The seminar started on Monday with a quick and tightly timed round of 2 minute personal introductions. Otherwise on Monday, Tuesday and Thursday the program was structured equally: in the morning we would have plenary presentations, which were interrupted ad-hoc by explorative discussions. These sessions were meant to bring everybody up-to-speed with ongoing and past initiatives. During and after lunch we used a board with sticky notes to define break-out groups. Each break-out group was centred around a specific discussion topic and (usually) a specific idea for an output document was associated with it. After coffee we would go back to the same break-out group to co-create the notes and lessons from each group (stored in a big shared online document). Between 17:00 and 18:00 we came back together and harvested the results of each breakout group with the others. People could and would freely switch between breakout groups but this was not a common thing.

On Wednesday we had an ``open-mic'' session with 8 presentations of around 10 minutes, sharing experiences and results, before we had a long walk in the surroundings. The organizers also designed an initial skeleton structure and ideas for the manifesto that day.

On Thursday afternoon and Friday morning we all worked together on our Dagstuhl Manifesto by first reworking our notes into the ideas around the manifesto, specifically a list of ``I will'' statements with references and motivation. Finally, Friday afternoon a small remaining group re-ordered the group's manifesto notes into a well structured list of 18 ``I will'' statements. Two of the organizers remained until Saturday morning to continue to edit the current report and the manifesto document.

\tableofcontents

\section{Overview of Talks}

%-------------------------------------------------------
%\newpage
\abstracttitle{The Netherlands eScience Center}
\abstractauthor[Rob van Nieuwpoort]{Rob van Nieuwpoort (The Netherlands eScience Center (NLeSC), NL)}

The Netherlands eScience Center (NLeSC) is the Dutch national hub for
the development and application of domain overarching software and
methods for the scientific community. NLeSC develops crucial bridges
between increasingly complex modern e-infrastructures and the growing
demands and ambitions of scientists from across all disciplines. The
application of digitally enhanced scientific practices makes sure that
return can be achieved from scientific investments. In support of this
goal NLeSC funds and simultaneously funds and participates in
multidisciplinary projects, with academia and industry, with optimized
data-handling, efficient computing and big-data analytics at their
core.

NLeSC contributes exclusively to multidisciplinary projects with the
potential to deliver scientific excellence, in terms of breakthroughs
and in the realization of unique eScience methodologies. Many
organizational practices, such as our open call strategy and other
funding models ensure that new projects fulfill these criteria.

Apart from contributions to scientific publications in high-impact
scientific journals and conferences, NLeSC’s primary deliverables are
eScience instruments (e.g., software tools, workflows). Whilst the
instruments may include a domain specific component, primarily these
tools overarch multiple domains. The instruments are efficient,
calibrated, reliable and accessible and based on excellent standards
of code quality utilizing meta-data standards and software development
environments. Successful instruments are made publicly available as
part of NLeSC’s eScience technology platform (eStep) program. This
platform provides easy access to the developed tools and instruments
to the broader scientific community and industry alike. NLeSC also
shares non-scientific technical results, documentation and best
practices in the knowledge base that also is a part of eStep.

NLeSC also plays a key role in optimizing and disseminating the best
practices in the areas of software sustainability and
data-stewardship, including the need to engage with communities of
practices, data-publishers and related initiatives.

The rapid growth of data and computing initiatives risks unnecessary
fragmentation and duplication. NLeSC works with numerous partner
organizations, nationally and internationally, to identify common
challenges such as training and career support for eScientists, as
well as providing thought leadership on issues such as
data-stewardship and software sustainability. NLeSC is a joint
initiative of the Dutch national research council (NWO) and the Dutch
organization for ICT in education and research (SURF).

%\license
%\jointwork{Bry, Fran\c{c}ois;}
%\abstractref[]{} % [] - URL, {} - reference description (a la thebibliography)
%\abstractrefurl{} % preferrably DOI-based URL

\abstracttitle{What We Have Learned About Using Software Engineering Practices in Scientific Software}
\abstractauthor[Jeffrey Carver]{Jeffrey Carver (University of Alabama, US)}
  
The increase in the importance of Scientific Software motivates the need to identify and understand which software engineering (SE) practices are appropriate. Because of the uniqueness of the scientific software domain, existing SE tools and techniques developed for the business/IT community are often not efficient or effective. Appropriate SE solutions must account for the salient characteristics of the scientific software development environment. To identify these solutions, members of the SE community must interact with members of the scientific software community. This presentation will discuss the findings from a series of case studies of scientific software projects, an ongoing workshop series, and the results of interactions between my research group and scientific software projects.  
  
\abstracttitle{Domain-specific modeling languages }  
\abstractauthor[Benoit Combemale]{Benoit Combemale (IRISA - Rennes, FR)}

Software languages are the primary way to ask the computer to solve a particular problem. With the increasing complexity of scientific problems to be solved by computer machines, a primary source of accidental complexity is the large gap between the high-level concepts used by scientists to express their problem statements and the low-level abstractions provided by general-purpose programming languages. 

Domain-specific (modeling) languages (DSMLs) can help closing the gap between the (problem) space in which scientists work and the implementation space (solution) in which programming is involved most of the time. Incorporating domain-specific concepts and high-quality development experience into DSMLs can significantly improve scientist productivity and experimentation quality. DSMLs can also support innovation within new problem spaces not already addressed by other software languages, as well as socio-technical coordination between scientists from different fields for solving global problems such as sustainability. 

\abstracttitle{}
\abstractauthor[]{Ralf La\"emmel}

\abstracttitle{Organising a research team around the research software around the research team in software engineering}
\abstractauthor[]{Jurgen J. Vinju (Centrum Wiskunde \& Informatica, NL)}

This was about motivation, experiences and lessons learned around the SWAT research group at CWI and its core product/platform/framework/language used for doing research and transfer. 

\abstracttitle{The OSSMETER platform}
\abstractauthor[]{Jurgen J. Vinju}

This briefly introduced the OSSMETER platform for monitoring and comparing open-source software projects in terms of code, activity, issues and community. This open-source project can be used to monitor and assess projects and may be applicable to the domain of scientific software as well.

\abstracttitle{ABC: Using Object Tracking to Automate Behavioural Coding}  
\abstractauthor[]{}
 
Video data of people interacting with devices contains rich information about human behaviour that can be used to design or improve user experience. As a first step, it must be interpreted – or coded – into a form that can be analyzed systematically. The coding process is currently performed manually, and it can be slow and difficult, and biased by subjectivity. This is particularly problematic when trying to obtain data that should be objective, such as the move- ments of a user in relation to a device. We describe Auto- mated Behavioural Coding (ABC), an open source object tracking technique designed to log user and device move- ments, and then output positional data that can be used to model interaction. We validate the technique in a study of dual screen TV viewing, and show that the ABC tool is able to correctly classify the direction of gaze to the TV or tablet up to 95\% of the time, in a fraction of the time it takes to capture this data manually.
 
\abstracttitle{A Short (and Probably Incomplete) History of Research Software Engineers in the UK}  
\abstractauthor[]{Robert Haines, University of Manchester, GB}

From pre-history to the present, via the SSI Collaborations Workshop in 2012, we describe dthe timeline of the development of the RSE job role in the UK.

\abstracttitle{Code Work in Science: How it changes, and Why it matters how we talk about change}
\abstractauthor[]{Katie Kuksenok (University of Washington - Seattle, US)} 

As software projects in science become more ambitious (and expensive in time, energy, and money), they increasingly require a language that recognizes and rewards the collective pursuit of uncertain possible worlds. I developed such a language, and I believe it would be useful to use it to articulate our current imagination, as a scholarly community, of the "perfect world" with regard to scientific programming. This articulation includes social and cognitive components of personalized "working environment." This allows me to include curiosity and excitement as much as panic and breakdown as causes, or opportunities, for small deliberate acts of change.

The conceptual framework is based on an 18-month interview and observation study of "code work" in science. "Code work," here, is used to include scripting, programming, software development, and other activities that span different projects but that demand common skills and aesthetics from the individuals undertaking them.

As attachments, I have included: (1) a LINK to a short blog post summary with figures (a good overview), (2) slides with a very short introduction, and (3) the final PDF of my dissertation (for the dedicated). Throughout the workshop, I intend to work on a reflection/summary of our discussion using the framework I have developed. In the Monday 2-minute introduction, I will introduce the framework, following the attached LINK (1). In the Wednesday open mic, I'll summarize in 10 minutes my "first-draft" take, which I would refine and contribute as part of the output document.

LINK (1): https://medium.com/hci-design-at-uw/code-work-in-science-how-it-changes-and-why-it-matters-how-we-talk-about-change-fecd33471b0
 
\abstracttitle{Software Metadata: Describing ``dark software'' in Geosciences}  
\abstractauthor[]{Daniel Garijo (Technical University of Madrid, ES)}

In this talk I provide an overview of the current state of the art for software description in geosciences, along with our approach to facilitate this task in OntoSoft, a distributed semantic registry for scientific software. Three key aspects of OntoSoft are: a software metadata ontology designed for scientists, a distributed approach to software registries that targets communities of interest, and metadata crowdsourcing through access control. Software metadata is organized using the OntoSoft ontology, designed to support scientists to share, document, and reuse software, and organized along six dimensions: identify software, understand and assess software, execute software, get support for the software, do research with the software, and update the software.

\abstracttitle{Restoring reproducibility: Making scientist software discoverable}
\abstractauthor[]{Alice Allen, University of Maryland (College Park, US)}

Taken from a presentation given at the National Library of Medicine (US), this Open Mic talk gives a quick overview of the Astrophysics Source Code Library (ASCL) and how it seeks to make research software used in astronomy more discoverable. It covers the changes in the ASCL over time and its impact.

The PowerPoint file has speaker notes that should mostly convey what I said in the presentation.

\abstracttitle{Lessons from the YT project}
\abstractauthor{Matthew Turk}

In this talk, I describe the engineering practices, both social and
technical, around the \href{yt project}{http://yt-project.org}.  I describe the
positive aspects and the failure modes, and how we have attempted to
route around these failure modes.

\abstracttitle{The Workshop on Sustainable Software for Science: Practice and Experiences (WSSSPE)}
\abstractauthor{Dan Katz (University of Illinois at Urbana Champaign, US)}

Progress in scientific research is dependent on the quality and accessibility of research software at all levels. It is now critical to address many new challenges related to the development, deployment, maintenance, and sustainability of open-use research software: the software upon which specific research results rely.  Open-use software means that the software is widely accessible (whether open source, shareware, or commercial).  Research software means that the choice of software is essential to specific research results; using different software could produce different results.

In addition, it is essential that scientists, researchers, and students are able to learn and adopt a new set of software-related skills and methodologies. Established researchers are already acquiring some of these skills, and in particular, a specialized class of software developers is emerging in academic environments who are an integral and embedded part of successful research teams. WSSSPE provides a forum for discussion of these challenges, including both positions and experiences, and a forum for the community to assemble and act.

This talk focuses on the Third Workshop on Sustainable Software for Science: Practice and Experiences (WSSSPE3). It summarized the discussions, future steps, organization, and status of a set of self-organized working groups on topics including developing pathways to funding scientific software; constructing useful common metrics for crediting software stakeholders; identifying principles for sustainable software engineering design; reaching out to research software organizations around the world; and building communities for software sustainability. Some of these groups have executed these activities that they scheduled, some have in part, and others have not.  A point of discussion is why these groups came to these points, and how the WSSSPE community can encourage groups to act. 

\abstracttitle{Sustainability Design}
\abstractauthor[]{Christoph Becker (University of Toronto, CA)}


\section{Research questions}

\section{Breakout sessions}

\end{document}
